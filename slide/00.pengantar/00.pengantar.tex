\documentclass[pdflatex,compress]{beamer}

%\usetheme[dark,framenumber,totalframenumber]{ElektroITK}
\usetheme[darktitle,framenumber,totalframenumber]{ElektroITK}

\usepackage{lipsum}

\title{PEMODELAN JARINGAN KOMUNIKASI}
\subtitle{Pengantar Pemodelan Jaringan Komunikasi}

\author{Mifta Nur Farid, S.T., M.T.}

\begin{document}

\maketitle

% ----------------------------------------------------------------------------

\section{Kontrak Perkuliahan}

\subsection{Capaian Pembelajaran Mata Kuliah}

\begin{frame}
	\frametitle{Capaian Pembelajaran Mata Kuliah}
	Mahasiswa mampu menganalisis model dan simulasi dari jaringan telekomunikasi
\end{frame}

\subsection{Bahan Kajian}

\begin{frame}
	\frametitle{Bahan Kajian}
	\begin{enumerate}
		\item Dasar Pemodelan dan Simulasi Jaringan
		\item M2M, D2D, model jaringan dan standar jaringan Adhoc IEEE 802.15.4 dan IEEE 802.11
		\item Desain jaringan dan Parameter Protokol
		\item Pemodelan Jaringan menggunakan model matematis
		\item Desain Jaringan dan Parameter Protokol Menggunakan Tool Software Network Simulator
	\end{enumerate}
\end{frame}

\subsection{Pustaka}

\begin{frame}
	\frametitle{Pustaka}
	\begin{enumerate}
		\item Issariyakul, T. \& Hossain, E. (2012). Introduction to Network Simulator NS2. New York: Springer.
		\item Law, A.M. \& Kelton, W. D. (2001). Simulation Modeling and Analysis. New York: McGraw-Hill.
		\item Altman, E. \& Jiménez, T. (2012). NS Network Simulator for Beginners. Berkeley: Morgan \& Claypool
		Publishers.
	\end{enumerate}
\end{frame}

\subsection{Jenis dan Bobot Evaluasi}

\begin{frame}
	\frametitle{Jenis dan Bobot Evaluasi}
	\begin{enumerate}
		\item Kehadiran: 10\%
		\item Tugas: 10\%
		\item Kuis: 15\%
		\item UTS: 20\%
		\item UAS: 20\%
		\item Tugas Besar: 25\%
	\end{enumerate}
\end{frame}
% ----------------------------------------------------------------------------

\section{Pengantar Pemodelan Jaringan Komunikasi}

\subsection{Pendahuluan}

\begin{frame}
	\frametitle{Pendahuluan}
\end{frame}

\end{document}
