\documentclass[pdflatex,compress]{beamer}

%\usetheme[dark,framenumber,totalframenumber]{ElektroITK}
\usetheme[darktitle,framenumber,totalframenumber]{ElektroITK}

\usepackage{lipsum}

\title{PEMODELAN JARINGAN KOMUNIKASI}
\subtitle{Pengantar Pemodelan Jaringan Komunikasi}

\author{Mifta Nur Farid, S.T., M.T.}

\begin{document}

\maketitle

% ----------------------------------------------------------------------------

\section{Kontrak Perkuliahan}

\subsection{Capaian Pembelajaran Mata Kuliah}

\begin{frame}
	\frametitle{Capaian Pembelajaran Mata Kuliah}
	Mahasiswa mampu menganalisis model dan simulasi dari jaringan komunikasi.
\end{frame}

\subsection{Bahan Kajian}

\begin{frame}
	\frametitle{Bahan Kajian}
	\begin{enumerate}
		\item Dasar Pemodelan dan Simulasi Jaringan
		\item M2M, D2D, model jaringan dan standar jaringan Adhoc IEEE 802.15.4 dan IEEE 802.11
		\item Desain jaringan dan Parameter Protokol
		\item Pemodelan Jaringan menggunakan model matematis
		\item Desain Jaringan dan Parameter Protokol Menggunakan Tool Software Network Simulator
	\end{enumerate}
\end{frame}

\subsection{Pustaka}

\begin{frame}
	\frametitle{Pustaka}
	\begin{enumerate}
		\item Issariyakul, T. \& Hossain, E. (2012). Introduction to Network Simulator NS2. New York: Springer.
		\item Law, A.M. \& Kelton, W. D. (2001). Simulation Modeling and Analysis. New York: McGraw-Hill.
		\item Altman, E. \& Jiménez, T. (2012). NS Network Simulator for Beginners. Berkeley: Morgan \& Claypool
		Publishers.
	\end{enumerate}
\end{frame}

\subsection{Jenis dan Bobot Evaluasi}

\begin{frame}
	\frametitle{Jenis dan Bobot Evaluasi}
	\begin{enumerate}
		\item Kehadiran: 10\%
		\item Tugas: 10\%
		\item Kuis: 15\%
		\item UTS: 20\%
		\item UAS: 20\%
		\item Tugas Besar: 25\%
	\end{enumerate}
\end{frame}
% ----------------------------------------------------------------------------

\section{Pengantar Pemodelan Jaringan Komunikasi}

\subsection{Pendahuluan}

\begin{frame}
	\frametitle{Pendahuluan}
	\begin{itemize}
		\item Manusia berkomunikasi dan saling bertukar informasi sepanjang waktu.
		\item Dalam beberapa dekade terakhir, banyak teknologi yang tercipta untuk membantu proses pertukaran informasi dengan cara yang kreatif dan efisien.
		\item Di antaranya adalah telepon, tv dan radio broadcasting, komputer dan internet, serta teknologi wireless.
		\item Awalnya, teknologi-teknologi tersebut ada dan beroperasi secara mandiri, melakukan tujuannya masing-masing.
	\end{itemize}
\end{frame}

\begin{frame}
	\frametitle{Pendahuluan}
	\begin{itemize}
		\item Kemudian baru-baru ini, teknologi-teknologi tersebut mulai manyatu, dan tidak bisa dipungkiri lagi bahwa hasilnya adalah jaringan telekomunikasi yang kita gunakan saat ini.
		\item Pada mata kuliah ini, jaringan komunikasi akan dijelaskan kembali. Begitu juga dengan model dan layer jaringannya.
		\item Kemudian cara untuk mendisain dan memodelkan sistem jaringan telekomunikasi yang kompleks akan diajarkan. Dilanjutkan dengan melakukan simulasi dari sistem tersebut menggunakan network simulator.
	\end{itemize}
\end{frame}

\subsection{Jaringan Komputer}

\begin{frame}
	\frametitle{Jaringan Komputer}
	\begin{itemize}
		\item Jaringan komputer/ computer network: sekumpulan interkoneksi antar komputer yang bertujuan untuk mengumpulkan, memproses, dan mendistribusikan informasi.
		\item Komputer: workstation, server, router, modems, base station, dan wireless extension point.
		\item Komputer-komputer terhubung dengan communication link: copper cable, fiber optic cable, dan microwave/satellite/radio link.
		\item Jaringan komputer $\rightarrow$ nested dan/atau interkoneksi dari beberapa jaringan $\rightarrow$ internet
		\item Internet: jaringan di dalam jaringan $\rightarrow$ puluhan ribu jaringan yang menginterkoneksikan/ menghubungkan jutaan komputer di seluruh dunia.
	\end{itemize}
\end{frame}

\subsection{Konsep Layering}

\begin{frame}
	\frametitle{Konsep Layering}
	
\end{frame}

\end{document}
